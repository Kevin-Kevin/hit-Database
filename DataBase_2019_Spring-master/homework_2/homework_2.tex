\documentclass[10pt,a4paper]{article}
	% 使用12pt(对应于中文的小四号字)
	% \usepackage{xeCJK} 
	% \setmainfont{Times New Roman} 
	% \setCJKmainfont[BoldFont=Hei]{Hei} 
	\usepackage{ctex}
	\usepackage{geometry}   %设置页边距的宏包
    \usepackage{titlesec}   %设置页眉页脚的宏包
    \usepackage{amssymb}
    \usepackage{amsmath}
    \usepackage{minted}
    \geometry{left=3.17cm,right=3.17cm,top=2.54cm,bottom=2.54cm}  %设置 上、左、下、右 页边距
	\begin{document}
	\newpagestyle{main}{            
		\sethead{哈尔滨工业大学}{数据库系统\_第二次作业}{1160300314 朱明彦}     %设置页眉
		% \setfoot{左页脚}{中页脚}{右页脚}      %设置页脚,可以在页脚添加 \thepage  显示页数
		\setfoot{}{}{\thepage}
		\headrule                                      % 添加页眉的下划线
		% \footrule                                       %添加页脚的下划线
	}
	\pagestyle{main}    %使用该style
	
	% \setlength{\baselineskip}{15.6pt}
	\setlength{\parskip}{0pt}
	\renewcommand{\baselinestretch}{1.5}
	
    \paragraph{一、}
    \begin{enumerate}
        \item[1)] 
        \begin{minted}{SQL}
            SELECT CNAME, TEACHER
            FROM S, C, SC
            WHERE S.S# = SC.S# AND C.C# = SC.C# AND S.S# = "003";
        \end{minted}
        \item[2)]\begin{minted}{SQL}
            SELECT DISTINCT SNAME
            FROM S, C, SC
            WHERE S.S# = SC.S# AND C.C# = SC.C# AND S.AGE = "男"
                 AND C.TEACHER = "程军";
        \end{minted}
        \item[3)]\begin{minted}{SQL}
            SELECT CNAME
            FROM C
            WHERE CNAME NOT IN(
                SELECT DISTINCT CNAME
                FROM S, C, SC
                WHERE S.S# = SC.S# AND C.C# = SC.C# AND S.SNAME = "刘丽";
            )
        \end{minted}
        \item[4)]\begin{minted}{SQL}
            SELECT S#, SNAME
            FROM S, SC
            WHERE S.S# = SC.S#
            GROUP BY S.S# HAVING AVG(GRADE) < 60;
        \end{minted}
        \textbf{注意此处的问题,在使用了\texttt{GROUP BY}子句后,\texttt{SELECT}子句的列名表中只能出现分组属性和聚集函数;从而,因为
        因为此处需要\texttt{S\#, SNAME},从而在分组时应该同时选择\texttt{S\#, SNAME},修改后的SQL语句如下:}
        \begin{minted}{SQL}
            SELECT S#, SNAME
            FROM S, SC
            WHERE S.S# = SC.S#
            GROUP BY S.S# S.SNAME HAVING AVG(GRADE) < 60;
        \end{minted}
        \item[5)]\begin{minted}{SQL}
            SELECT S#
            FROM S, SC
            WHERE S.S# = SC.S#
            GROUP BY S.S# HAVING COUNT(*) >= 3
            ORDER BY S# ASC;
        \end{minted}
    \end{enumerate}
    
    \paragraph{二、}
    \begin{enumerate}
        \item[1)]\begin{minted}{SQL}
            CREATE TABLE Classes
            (
                class CHAR(20) PRIMARY KEY,
                type CHAR(2) NOT NULL,
                country CHAR(15),
                numGuns SMALLINT,
                bore, SMALLINT,
                displacement, INT
            );
        \end{minted}
        
        \begin{minted}{SQL}
            CREATE TABLE Ships
            (
                name CHAR(20) PRIMARY KEY,
                class CHAR(20),
                launched SMALLINT,
                FOREIGN KEY (class) REFERENCES Classes(class)
            );
        \end{minted}

        注意
        \begin{itemize}
            \item 在插入\texttt{Classes}时,主键\texttt{class}可以唯一区分实体;
            \item 在插入\texttt{Ships}时,主键\texttt{name}可以唯一区分实体,并且外键\texttt{class}需要存在。
        \end{itemize}
        \item[2)]\begin{itemize}
            \item[a.]\begin{minted}{SQL}
                INSERT 
                INTO Classes
                VALUES("Nelson", "bb", "Gt.Britain", 9, 16, 34000);

                INSERT
                INTO Ships
                VALUES("Nelson", "Nelson", 1927);

                INSERT
                INTO Ships
                VALUES("Rodney", "Nelson", 1927);
            \end{minted}
            \item[b.]\begin{minted}{SQL}
                DELETE
                FROM Ships
                WHERE name IN
                (
                    SELECT
                    FROM Ships, Outcomes
                    WHERE Ships.name = Outcomes.ship
                        AND Outcomes.result = "sunk"
                );
            \end{minted}
            \item[c.]\begin{minted}{SQL}
                UPDATE Classes
                SET bore = bore * 2.5;

                UPDATE Classes
                SET displacement = displacement / 1.1;
            \end{minted}  
        \end{itemize}
        \item[3)]\begin{minted}{SQL}
            CREATE VIEW (class, type, numGuns, bore, displacement, launched)
            AS
            SELECT class, type, numGuns, bore, displacement, launched
            FROM Classes, Ships
            WHERE Classes.class = Ships.ship
                AND country = "Gt.Britain";
        \end{minted}
        \item[4)]\begin{minted}{SQL}
            SELECT battle
            FROM Ships, Outcomes
            WHERE Ships.name = Outcomes.ship
                AND Ships.class = "Kongo";
        \end{minted}
        \item[5)]\begin{minted}{SQL}
            SELECT AVG(numGuns)
            FROM Classes
            WHERE type = "bb";
        \end{minted}
        \textbf{注意此处的问题应该是,所有的主力舰的火炮数量加和然后平均,改正后的SQL语句如下:}
        \begin{minted}{SQL}
            SELECT AVG(numGuns)
            FROM Classes, Ships
            WHERE Classes.class = Ships.class AND Classes.type = "bb";
        \end{minted}
        \item[6)]\begin{minted}{SQL}
            SELECT  MIN(launched)
            FROM Classes, Ships
            WHERE Classes.class = Ships.class
            GROUP BY Classes.class;
        \end{minted}
        \textbf{注意需要同时显示类名和下水年份,修改后的SQL语句如下:}
        \begin{minted}{SQL}
            SELECT  Classes.class, MIN(launched)
            FROM Classes, Ships
            WHERE Classes.class = Ships.class
            GROUP BY Classes.class;
        \end{minted}
        \item[7)]
        \textbf{下面这种写法存在问题,对于没有船沉没但是有超过3艘船的类型,没有输出!}
        \begin{minted}{SQL}
            SELECT class, COUNT(result)
            FROM Ships, Outcomes
            WHERE Ships.name = Outcomes.ship
                AND Outcomes.result = "sunk"
                AND class IN
                (
                    SELECT class
                    FROM Ships, Outcomes
                    WHERE Ships.name = Outcomes.ship
                        GROUP BY class HAVING COUNT(*) >= 3
                )
                GROUP BY class;
        \end{minted}
		\textbf{应该改为如下写法,注意不使用where过滤是否沉船条件:}
		\begin{minted}{SQL}
            SELECT class, SUM(case 1 when Outcomes.result = "sunk" else 0 end)
            FROM Ships, Outcomes
            WHERE Ships.name = Outcomes.ship
                AND class IN
                (
                    SELECT class
                    FROM Ships, Outcomes
                    WHERE Ships.name = Outcomes.ship
                        GROUP BY class HAVING COUNT(*) >= 3
                )
                GROUP BY class;
        \end{minted}
        \item[8)]\begin{minted}{SQL}
            SELECT name
            FROM Ships
            UNION
            SELECT ship
            FROM Outcomes;
        \end{minted}
        \item[9)]\begin{minted}{SQL}
            SELECT country
            FROM Classes
            GROUP BY country HAVING COUNT(type) >= 2;
        \end{minted}
        \item[10)]\begin{minted}{SQL}
            SELECT country
            FROM Classes
            WHERE numGuns >= ALL
                (
                    SELECT numGuns
                    FROM Classes
                );
        \end{minted}
        \item[11)]\begin{minted}{SQL}
            SELECT Classes.class
            FROM Classes, Ships, Outcomes
            WHERE Classes.class = Ships.class
                AND Ships.name = Outcomes.ship
                AND Outcomes.result = "sunk"
            GROUP BY Classes.class HAVING COUNT(*) >= 1;
        \end{minted}
    \end{enumerate}
    \paragraph{三、}
    \begin{enumerate}
        \item[1)]\begin{minted}{SQL}
            CREATE TABLE Orders
            (
                OrderID CHAR(20) PRIMARY KEY,
                SupplierID CHAR(20), 
                MovieID CHAR(20),
                Copies INT,
                FOREIGN KEY (SupplierID) REFERENCES Suppliers(SupplierID),
                FOREIGN KEY (MovieID) REFERENCES Movies(MovieID)
            );

            CREATE TABLE Rentals
            (
                CustomerID CHAR(20),
                TapeID CHAR(20),
                CkoutDate DATE,
                Duration INT,
                PRIMARY KEY (CustomerID, TapeID, CkoutDate),
                FOREIGN KEY (CustomerID) REFERENCES Customers(CustomerID),
                FOREIGN KEY (TapeID) REFERENCES Inventory(TapeID)
            );
        \end{minted}
        \item[2)]\begin{enumerate}
            \item[a.]\begin{minted}{SQL}
                INSERT
                INTO Rentals
                VALUES("9823", "5600", 2017-03-26, 30);
            \end{minted}
        \end{enumerate}
            \item[b.]\begin{minted}{SQL}
                DELETE
                FROM Rentals
                WHERE CkoutDate < '2000-01-01';
            \end{minted}
            \item[c.]\begin{minted}{SQL}
                UPDATE MovieSupplier
                SET Price = Price / 6.88;
            \end{minted}
        \item[3)]\begin{minted}{SQL}
            CREATE VIEW JHV_Suppliers (MovieName, Price)
            AS
            SELECT MovieName, Price
            FROM Movies, Suppliers, MovieSupplier
            WHERE Movies.MovieID = MovieSupplier.MovieID
                AND Suppliers.SupplierID = MovieSupplier.SupplierID
                AND Suppliers.SupplierName = "Joe's House of Video";
        \end{minted}
        \item[4)]\begin{minted}{SQL}
            SELECT SupplierName, COUNT(MovieID)
            FROM Inventory, Suppliers, MovieSupplier
            WHERE Inventory.MovieID = MovieSupplier.MovieID
                AND Suppliers.SupplierID = MovieSupplier.SupplierID
            GROUP BY Suppliers.SupplierID;
        \end{minted}
        \item[5)]\begin{minted}{SQL}
            SELECT MovieName
            FROM Orders, Movies
            WHERE Movies.MovieID = Orders.MovieID
                AND SUM(Orders.Copies) > 5
            GROUP BY Orders.MovieID;
        \end{minted}
        \item[6)]\begin{minted}{SQL}
            SELECT MovieName
            FROM Inventory, Movies
            WHERE Movies.MovieID = Inventory.MovieID
            GROUP BY Movies.MovieID HAVING COUNT(TapeID) > 1;
        \end{minted}
        \item[7)]\begin{minted}{SQL}
            SELECT MovieName
            FROM Movies, Inventory, Rentals
            WHERE Movies.MovieID = Inventory.MovieID
                AND Rentals.TapeID = Inventory.TapeID
                AND Rentals.Duration >= ALL 
                (
                    SELECT Duration
                    FROM Rentals
                );
        \end{minted}
        \item[8)]\begin{minted}{SQL}
            SELECT DISTINCT MovieName
            FROM Movies
            WHERE MovieName NOT IN(
                SELECT DISTINCT MovieName
                FROM Movies, Inventory
                WHERE Movies.MovieID = Inventory.MovieID;
            )
        \end{minted}
        \item[9)]\begin{minted}{SQL}
            SELECT SupplierID, SupplierName
            FROM Suppliers, MovieSupplier
            WHERE Suppliers.SupplierID = MovieSupplier.MovieID
                AND MovieSupplier.MovieID IN
                (
                    SELECT MovieID
                    FROM Movies
                    WHERE MovieName = "Hacksaw Ridge"
                )
                AND MovieSupplier.Price <= ALL
                (
                    SELECT Price
                    FROM Movies, MovieSupplier
                    WHERE Movies.MovieID = MovieSupplier.MovieID
                        AND Movies.MovieName = "Hacksaw Ridge"
                );
        \end{minted}
        \item[10)]\begin{minted}{SQL}
            SELECT CustomerName
            FROM Movies, Rentals, Inventory, Customers
            WHERE Movies.MovieID = Inventory.MovieID
                AND Rentals.TapeID = Inventory.TapeID
                AND Customers.CustomerID = Rentals.CustomerID
                AND Movies.MovieName = "Beauty and the Beast"
            UNION
            SELECT CustomerName
            FROM Rentals, Inventory, MovieSupplier, Suppliers, Customers
            WHERE Rentals.CustomerID = Customers.CustomerID
                AND Inventory.TapeID = Rentals.TapeID
                AND MovieSupplier.SupplierID = Suppliers.SupplierID
                AND Inventory.MovieID = MovieSupplier.MovieID
                AND Inventory.MovieID IN
                (
                    SELECT MovieID
                    FROM MovieSupplier, Suppliers
                    WHERE MovieSupplier.SupplierID = Suppliers.SupplierID
                        AND Suppliers.SupplierName = "VWS Video"
                );
        \end{minted}
    \end{enumerate}
    

\end{document}